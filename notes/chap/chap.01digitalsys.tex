\ifx\mainclass\undefined
\documentclass[cn,11pt,chinese,black,simple]{../elegantbook}
% 微分号
\newcommand{\dd}[1]{\mathrm{d}#1}
\newcommand{\pp}[1]{\partial{}#1}

% FT LT ZT
\newcommand{\ft}[1]{\mathscr{F}[#1]}
\newcommand{\fta}{\xrightarrow{\mathscr{F}}}
\newcommand{\lt}[1]{\mathscr{L}[#1]}
\newcommand{\lta}{\xrightarrow{\mathscr{L}}}
\newcommand{\zt}[1]{\mathscr{Z}[#1]}
\newcommand{\zta}{\xrightarrow{\mathscr{Z}}}

% 积分求和号

\newcommand{\dsum}{\displaystyle\sum}
\newcommand{\aint}{\int_{-\infty}^{+\infty}}

% 简易图片插入
\newcommand{\qfig}[3][nolabel]{
  \begin{figure}[!htb]
      \centering
      \includegraphics[width=0.6\textwidth]{#2}
      \caption{#3}
      \label{\chapname :#1}
  \end{figure}
}

% 表格
\renewcommand\arraystretch{1.5}

% 日期

\newcommand{\homep}[1]{\section*{Problem #1}}
\begin{document}
\fi 
\def\chapname{01digitalsys}

% Start Here
\chapter{概述}

\section{数字系统概述}

数字系统涵盖方面广,

\begin{itemize}
    \item 处理器
    \item 存储器
    \item SoC: System on Chip
\end{itemize}

数字系统可以看作是一个微处理器外加一个交互接口。一般来说,数字系统包括算数逻辑单元,存储单元以及控制单元,硅集成后就可以成为 IC 。



半导体产品包括:

\begin{itemize}
    \item 光电器件
    \item 传感器
    \item 分立器件
    \item 集成电路
\end{itemize}

\subsection{微处理器}

微处理器几乎是数字系统的核心。

常见的微处理器有中央处理器即 CPU 。其中的控制单元对指令进行处理,使得存储单元以及运算单元与内存进行交互完成工作。通过指令集对底层进行交互。

另一种是图形处理器即 GPU ,控制单元相对较少,但是有更多的细粒度 ALU 以及更大的显存。

还有现场可编程逻辑门阵列即 FPGA ,通过逻辑设计对开关核进行不同的连线,实现不同的逻辑功能。

专用处理器即 ASIC ,为特定的系统的需要而设计,速度快,但是设计成本更高。

其他芯片还有: DSP 数字信号处理器,ISP 图像信号处理器, MCU 微处理器, SoC 系统级芯片或片上系统。

\section{数字逻辑电路}

数字逻辑电路实现了数字信号逻辑运算的电路,实现离散值的逻辑计算。

组合逻辑电路的输入输出字号见没有反馈延时通路,且不含存储单元。

时序逻辑电路由组合电路以及存储单元组成,存在反馈电路。可以分为同步时序电路以及异步时序电路。

最早的逻辑器件有真空电子管以及晶体管。之后出现了金属氧化物半导体场效应管也就是 MOSFET 。

\section{逻辑}

什么是逻辑?可以用 0 与 1 表达逻辑与任务。香农提到,逻辑可以使用逻辑电路实现。通过逻辑器件实现逻辑,综合成逻辑门后设计成逻辑电路最终实现到处理器。

\subsection{数字与模拟信号}

数字信号

\begin{itemize}
    \item 时域离散
    \item 值域为悠闲地集合
    \item 来自物理世界
    \item 热波动噪声
\end{itemize}

模拟信号:

\begin{itemize}
    \item 时域连续
    \item 值域连续
    \item 来自布尔相关方程
    \item 采样噪声
\end{itemize}


由于模拟信号的噪声多且难以消除,存在累积效应,更多采用数字信号。

\subsection{数字信号的获得}

通过采样将模拟信号转换为数字信号,效果与采样频率和采样精度有关。


\section{数字电路设计方法学}

\subsection{设计方法论}

存在三个设计域:

\begin{itemize}
    \item 行为域:Spec, Algorithm, RTL, Bool, Differential Equation.
    \item 结构域: Transistor, Gate, ALU/MUX/Reg, Processor/Subsystem, CPU/Mem/SoC 
    \item 几何域: Rectangle Polygon-group, Standard cell, Macro cell, Block/Chip, Chip/Board
\end{itemize}

\subsection{层次化设计}

自底向上设计(Bottom-Up) :缺乏全局规划,迭代优化难度大。

自顶向下设计(Top-Down) : 通过“设计-验证-修改”的反复迭代,最终得到满足性能以及功能要求的结果。

混合设计模式:从上到下设计,从下到上实现。

\subsection{数字系统的实现技术}

按照分类方式分:

\begin{itemize}
    \item 全定制:搭积木式设计,几乎没有灵活性且成本很高
    \item 半定制:标准单元式,标准单元库有助于提高布图效率,自动化程度高,设计周期短,ASIC 广泛使用
    \item 可编程:灵活度高,周期短,上市快,功耗大,成本高,速度慢
\end{itemize}

\section{数字系统设计自动化}

设计芯片的方法从手工设计、计算机辅助,走向了电子设计自动化( EDA )。EDA 以计算机与电子技术为先导,汇集了计算机图形学、拓扑学等知识。

CAD 进行独立的 EDA 使用,辅助版图设计,PCB 设计,电路模拟等。

CAE 实现了设计工作的集成话,原理图输入,逻辑仿真,自动布局,功能模拟以及分析验证可以系统化进行。

EDA 自顶向下的设计可以将经历集中在方案以及架构创新上。

EDA 是集成电路的产业龙头,推动封装测试行业从二维转向三维,实现了 PCB 的板级系统的硅上互联,时间摩尔转向空间摩尔。

分类大致有

\begin{itemize}
    \item 电子电路设计:HSPICE, SPECTRE
    \item PCB:Protel
    \item PLD\footnote{Programmable Logic Device}:Quaturs II,ISE
    \item IC: ModelSim
    \item ASIC:Candence,Synopsys,Mentor
\end{itemize}


系统设计流程分为 

\begin{itemize}
    \item 前端设计
    \begin{itemize}
        \item 系统功能设计
        \item 系统结构划分
        \item 电路 / RTL 
        \item 逻辑综合
    \end{itemize}
    \item 生成门级网表
    \item 后端设计
    \begin{itemize}
        \item 物理版图设计
        \item 物理版图验证
        \item 核签
        \item 流片
    \end{itemize}
\end{itemize}

\begin{table}[htb]
    \centering
    \caption{层次化设计}
    \begin{tabular}{llll}
    \hline
          & 时序单位 & 基本单元   & 功能表述 \\ \hline
    系统级设计 & 数据处理 & 进程 与通信 & 自然语言 \\
          &      &        &      \\
          &      &        &      \\
          & up   & down   &      \\ \hline
    \end{tabular}
    \end{table}

什么是综合(Synthesis),人工的设计称为设计,自动化设计称为综合。

\begin{definition}[综合]
    从较高层次的设计描述到较低层次的转换、映射并进行一定的优化设计的过程。
\end{definition}

高层综合从高级语言直接转换,涉及到多目标的优化,如资源延时等。

逻辑综合将 RTL 级代码转换为基于标准单元库满足约束的门级网表。



% End Here

\let\chapname\undefined
\ifx\mainclass\undefined
\end{document}
\fi 
\documentclass[lang=cn,11pt,a4paper,cite=authoryear]{elegantpaper}

% 微分号
\newcommand{\dd}[1]{\mathrm{d}#1}
\newcommand{\pp}[1]{\partial{}#1}

% FT LT ZT
\newcommand{\ft}[1]{\mathscr{F}[#1]}
\newcommand{\fta}{\xrightarrow{\mathscr{F}}}
\newcommand{\lt}[1]{\mathscr{L}[#1]}
\newcommand{\lta}{\xrightarrow{\mathscr{L}}}
\newcommand{\zt}[1]{\mathscr{Z}[#1]}
\newcommand{\zta}{\xrightarrow{\mathscr{Z}}}

% 积分求和号

\newcommand{\dsum}{\displaystyle\sum}
\newcommand{\aint}{\int_{-\infty}^{+\infty}}

% 简易图片插入
\newcommand{\qfig}[3][nolabel]{
  \begin{figure}[!htb]
      \centering
      \includegraphics[width=0.6\textwidth]{#2}
      \caption{#3}
      \label{\chapname :#1}
  \end{figure}
}

% 表格
\renewcommand\arraystretch{1.5}

% 日期

\newcommand{\homep}[1]{\section*{Problem #1}}

\title{数字电路高层次综合设计\quad 第十周作业}
\author{范云潜 18373486}
\institute{微电子学院 184111 班}
\date{\zhtoday}

\begin{document}

\maketitle

\tableofcontents

\listoffigures

\section*{注意}

本次实验的源代码在压缩包的 \lstinline{./homework10/} 下,并且已经按照实验序号标号,如 \lstinline{./homework10/01mux} 代表第一个实验。

\section{二路选择器}

本次核心的写法是 \lstinline{assign out = flag ? ans_1 : ans_0} 。


\begin{figure}
    \centering
    \caption{二路选择器结果}\label{01} 
    % line 1
    \sfig[0.45]{hw10021.jpg} \hfill 
    \sfig[0.45]{hw10022.jpg}
\end{figure}

\section{16 进制显示}

本次核心写法是 \lstinline{case()...default: } ,对不同的输入进行编码即可,若是从共阴极转换到共阳极只需要将 \lstinline{assign out = ~out_pre} 。


\begin{figure}
    \centering
    \caption{16 进制显示结果}\label{02} 
    % line 1
    \sfig[0.23]{hw10031.jpg} \hfill 
    \sfig[0.23]{hw10032.jpg} \hfill 
    \sfig[0.23]{hw10033.jpg} \hfill 
    \sfig[0.23]{hw10034.jpg} 
\end{figure}

\section{10 进制显示}

将上一实验的 \lstinline{case} 大于 \lstinline{9} 的部分取消即可。

\begin{figure}
    \centering
    \caption{10 进制显示结果}\label{03} 
    % line 1
    \sfig[0.3]{hw10041.jpg} \hfill 
    \sfig[0.3]{hw10042.jpg} \hfill 
    \sfig[0.3]{hw10043.jpg}  
    % \sfig[0.23]{hw10034.jpg} 
\end{figure}


\section{多位 10 进制显示}

实例化上一实验的接口即可。


\section{BCD 显示}

实例化 BCD 模块,并且将 BCD 的输出绑定到显示模块,同时忽略多位显示的最高位即可。

\begin{figure}
    \centering
    \caption{BCD 码显示结果}\label{03} 
    % line 1
    \sfig[0.3]{hw10051.jpg} \hfill 
    \sfig[0.3]{hw10052.jpg} \hfill 
    \sfig[0.3]{hw10053.jpg}  
    % \sfig[0.23]{hw10034.jpg} 
\end{figure}

\section{运算器}

核心写法是 \lstinline{assign out = flag ? ans_1 : ans_0} ,并且将输出绑定到显示模块。

\begin{figure}
    \centering
    \caption{运算器结果}\label{05} 
    % line 1
    \sfig{hw10011.jpg}
    \sfig{hw10012.jpg}
    \sfig{hw10013.jpg}
    \bigskip
    \bigskip
    \sfig{hw10014.jpg}
    \sfig{hw10015.jpg}
\end{figure}
% Start Here

% End Here

\end{document}
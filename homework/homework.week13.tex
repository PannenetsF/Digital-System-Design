\documentclass[lang=cn,11pt,a4paper,cite=authoryear]{elegantpaper}

% 微分号
\newcommand{\dd}[1]{\mathrm{d}#1}
\newcommand{\pp}[1]{\partial{}#1}

% FT LT ZT
\newcommand{\ft}[1]{\mathscr{F}[#1]}
\newcommand{\fta}{\xrightarrow{\mathscr{F}}}
\newcommand{\lt}[1]{\mathscr{L}[#1]}
\newcommand{\lta}{\xrightarrow{\mathscr{L}}}
\newcommand{\zt}[1]{\mathscr{Z}[#1]}
\newcommand{\zta}{\xrightarrow{\mathscr{Z}}}

% 积分求和号

\newcommand{\dsum}{\displaystyle\sum}
\newcommand{\aint}{\int_{-\infty}^{+\infty}}

% 简易图片插入
\newcommand{\qfig}[3][nolabel]{
  \begin{figure}[!htb]
      \centering
      \includegraphics[width=0.6\textwidth]{#2}
      \caption{#3}
      \label{\chapname :#1}
  \end{figure}
}

% 表格
\renewcommand\arraystretch{1.5}

% 日期

\newcommand{\homep}[1]{\section*{Problem #1}}

\title{数字电路高层次综合设计\quad 第十三周作业}
\author{范云潜 18373486}
\institute{微电子学院 184111 班}
\date{\zhtoday}

\begin{document}

\maketitle

\tableofcontents

\section{序列检测}

本任务的功能模块已经在前序课程中完成,因此只需要进行实例化即可。将 \lstinline{clk} 绑定到按键上,将 \lstinline{reset} 和 \lstinline{push} 绑定到开关,将输出绑定到 LED 灯。

综合结果如 \figref{01} 。 \qfig[01]{1201.png}{序列检测综合报告}

\section{交通信号灯}

时钟需要进行分频,调用之前的分频器模块,产生 5 Hz 的信号。

由于前序课程中的交通信号灯未曾预留倒计时的接口,因此需要进行一定修改。首先,在复位后,存在一定的缓冲时间,此时将对应的状态 \lstinline{idle} 作为信号 \lstinline{!on} 输出到倒计时模块。

倒计时模块按照实例化时的时间设定和初始状态设定进行工作,在接受到 \lstinline{on} 后和灯的闪烁保持同步。

对于倒计时模块输出的倒计时信号,需要通过 BCD 编码后将十位与各位输出到数码管显示模块。

综合结果如 \figref{02} 。 \qfig[02]{1202.png}{交通灯综合报告}

\section{流水灯}

将控制信号绑定到开关与按键后,将时钟绑定到分频器。关于流水灯的流水效果,可以通过 MASK 和移位操作便捷的完成:\lstinline{LED = MASK1 | MASK2}

\begin{enumerate}
    \item \lstinline{MASK1 = 1000 0000 , MASK2 = 0 } 对 \lstinline{MASK1} 进行右移位
    \item \lstinline{MASK1 = 1000 0000 , MASK2 = 0100 0000 } 对 \lstinline{MASK1, MASK2} 进行右移位
    \item \lstinline{MASK1 = 1000 0000 , MASK2 = 0000 0001 } 对 \lstinline{MASK1, MASK2} 进行反向移位
\end{enumerate}

综合结果如 \figref{03} 。 \qfig[03]{1203.png}{流水灯综合报告}

\section{计价器}

将时钟绑定到分频器,控制信号绑定到开关。核心是计数的转移,在进行计数的同时进行状态的转移,不同的状态每个公里数增加的金额是不一致的即可。之后将计数和里程通过 BCD 编码后输出到数码管模块。

综合结果如 \figref{04} 。 \qfig[04]{1204.png}{出租车综合报告}
% Start Here

% End Here

\end{document}
\documentclass[lang=cn,11pt,a4paper,cite=authoryear]{elegantpaper}

\input{needed.tex}

\title{数字电路高层次综合设计\quad 第十四周作业}
\author{范云潜 18373486}
\institute{微电子学院 184111 班}
\date{\zhtoday}

\begin{document}

\maketitle

% 作业内容:

\tableofcontents

\section{设计分析}

首先根据功能,分析所需要的功能模块:

\begin{itemize}
    \item 时钟模块 \lstinline{clock_hms} ,需要可以在特定情况下设置时间
    \item 闹钟模块 \lstinline{alarm} ,需要在时钟模块运行到特定时间时,出现特定的显示,持续到用户按键停止
    \item 秒表模块 \lstinline{timer} ,进行计时,需要有开始,暂停,归位的功能
\end{itemize}

为了实现这些功能模块的对外输出,需要的伺服模块有:

\begin{itemize}
    \item 分频器 \lstinline{tiks} 
    \item 基于模式选择的多路选择器 \lstinline{display_choose},用于将不同模块的信号输入到显示模块中,
    \item 显示模块 \lstinline{display_time} ,将之前的七段管显示 \lstinline{hex} 与 BCD 编码 \lstinline{bcd}合并在一起,对输入的时间进行显示
\end{itemize} 

此外需要核心的控制模块设计 \lstinline{ctrl} 。

\section{模块设计}

为了降低模块的复杂度,将所有的设置的控制部分转移到控制模块,各个功能模块仅采用控制模块给出的数值即可。各个功能模块中进行设置之前需要进行使能。

\subsection{时钟模块}

在除去复位( reset ) 以及置位( set )外,本模块需要一直进行计时,这样的目的是为闹钟提供参考时间,并且符合电子表的使用常识。

状态机仅有\textbf{工作}与\textbf{设置}两个状态,在设置时,时间直接转到控制模块给出的需要设定的时间。


\subsection{闹钟模块}

状态机仅有\textbf{工作}与\textbf{设置}两个状态,在设置时,时间直接转到控制模块给出的需要设定的时间。

在\textbf{工作}状态时,当设定时间到达,将输出的控制信号设定为\textbf{响铃},直到外部按下关闭闹铃键。

\subsection{秒表模块}

状态机只有\textbf{暂停}与\textbf{工作}两个状态。

\subsection{时间显示模块}

将 BCD 编码与七段管显示合并在一起,显示只需要给出时分秒各位数据即可。

\subsection{闹钟显示模块}

闹钟响铃以一位的数码管显示,响铃时产生动态效果。

\subsection{显示选择器}

用于选择各个功能模块的显示输出以及设置时的临时显示。


% Start Here

% End Here

\end{document}
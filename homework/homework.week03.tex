\documentclass[lang=cn,11pt,a4paper,cite=authoryear]{elegantpaper}

% 微分号
\newcommand{\dd}[1]{\mathrm{d}#1}
\newcommand{\pp}[1]{\partial{}#1}

% FT LT ZT
\newcommand{\ft}[1]{\mathscr{F}[#1]}
\newcommand{\fta}{\xrightarrow{\mathscr{F}}}
\newcommand{\lt}[1]{\mathscr{L}[#1]}
\newcommand{\lta}{\xrightarrow{\mathscr{L}}}
\newcommand{\zt}[1]{\mathscr{Z}[#1]}
\newcommand{\zta}{\xrightarrow{\mathscr{Z}}}

% 积分求和号

\newcommand{\dsum}{\displaystyle\sum}
\newcommand{\aint}{\int_{-\infty}^{+\infty}}

% 简易图片插入
\newcommand{\qfig}[3][nolabel]{
  \begin{figure}[!htb]
      \centering
      \includegraphics[width=0.6\textwidth]{#2}
      \caption{#3}
      \label{\chapname :#1}
  \end{figure}
}

% 表格
\renewcommand\arraystretch{1.5}

% 日期

\newcommand{\homep}[1]{\section*{Problem #1}}

\title{数字电路高层次综合设计\quad 第三周作业}
\author{范云潜 18373486}
\institute{微电子学院 184111 班}
\date{\zhtoday}

\begin{document}

\maketitle

作业内容:

% \tableofcontents

% Start Here
\section{设计与测试综述}

本节对实验的设计方法和测试方法进行说明,在具体模块的实现后不再赘述。

\subsection{设计方法}

对于结构级设计方法或者说从门级进行设计的方法,产生的硬件有着最高的确定性,但是设计较困难。在本实验设计的简单逻辑中仍可接受,若是与复杂计算逻辑相关,本这样的方法很难迅速的实现,并且最终结果也缺乏可读性,难以修改。

对于数据流式的设计方法,设计思路直观,设计迅速,但是底层实现不能确定,和工艺以及目标平台相关。这是在较高层次的设计中比较常用的方式。

在本次实验中仅涉及到组合逻辑,我们对于输入的下一节点进行分组,可以得到不同节点的计算式,重复应用直到输出级即可完成设计。

\subsection{测试方法}

对于所有的模块的 TestBench 均采用 Golden Model 方式进行测试,通过 Python 生成多组随机数据作为输入,计算后得到输出,将输入输出写入文件。在 TestBench 中通过 \lstinline{readmemh/readmemb} 读取到测试 Buf 中,逐个输入后比对输出,若是输出错误将信息 \lstinline{display} 到标准输出便于调试。同时,将所有的变量的波形进行存储,便于调试。

对于同一模块不同实现可以使用同样的,通过 \lstinline{`define, `ifdef} 等实例化不同的模块。

\section{7458 芯片}

按照芯片的门级表示进行门级元件调用或者通过 \lstinline{assign} 或 \lstinline{always} 进行两种方式的实现。

\section{向量逆序输出}

通过 Verilog 的位选语法进行分组与重组。

\section{加法器}

对于 16 位加法器通过 \lstinline{assign} 完成数据加法,输出到 \lstinline{cout,sum}

对于 32 位的加法器,需要由 16 位的加法器进行串联,第一级的进位为第二级的输入,进行模块调用即可。

% End Here

\end{document}
\documentclass[lang=cn,11pt,a4paper,cite=authoryear]{elegantpaper}

% 微分号
\newcommand{\dd}[1]{\mathrm{d}#1}
\newcommand{\pp}[1]{\partial{}#1}

% FT LT ZT
\newcommand{\ft}[1]{\mathscr{F}[#1]}
\newcommand{\fta}{\xrightarrow{\mathscr{F}}}
\newcommand{\lt}[1]{\mathscr{L}[#1]}
\newcommand{\lta}{\xrightarrow{\mathscr{L}}}
\newcommand{\zt}[1]{\mathscr{Z}[#1]}
\newcommand{\zta}{\xrightarrow{\mathscr{Z}}}

% 积分求和号

\newcommand{\dsum}{\displaystyle\sum}
\newcommand{\aint}{\int_{-\infty}^{+\infty}}

% 简易图片插入
\newcommand{\qfig}[3][nolabel]{
  \begin{figure}[!htb]
      \centering
      \includegraphics[width=0.6\textwidth]{#2}
      \caption{#3}
      \label{\chapname :#1}
  \end{figure}
}

% 表格
\renewcommand\arraystretch{1.5}

% 日期

\newcommand{\homep}[1]{\section*{Problem #1}}

\title{数字电路高层次综合设计\quad 第一周作业}
\author{范云潜 18373486}
\institute{微电子学院 184111 班}
\date{\zhtoday}

\begin{document}

\maketitle

作业内容:

\begin{itemize}
    \item 数字系统的实现方式有哪些?各有什么优缺点?
    \item 简述“片上系统(System on Chip, SOC)”
    \item 什么是Top-Down设计模式?优缺点是什么?
    \item 数字系统设计过程中,“设计说明书(Specification, Spec)”的作用是什么?通常包含哪些内容?
\end{itemize}

% \tableofcontents

% Start Here
\homep{数字系统的实现方式有哪些?各有什么优缺点?} 

按照设计方式分:

\begin{itemize}
    \item 全定制:设计人员完成所有晶体管和互连线的详细版图,以标准逻辑器件以及外围电路组成,一般具有很高的性能,根据提供的接口进行搭积木式设计,几乎没有灵活性且成本以及复杂度很高
    \item 半定制:标准单元式设计,标准单元库如们阵列与标准模块等有助于提高布局布线效率,自动化程度高,设计周期短,ASIC 广泛使用
    \item 可编程:基于 FPGA 等可编程逻辑器件,灵活度高,周期短,上市快,但是存在功耗大,成本高,速度慢的问题
\end{itemize}

\homep{简述“片上系统(System on Chip, SOC)”} 

SoC 是将一种电子系统集成到单一的芯片上的一种方式,一般具有 

\begin{itemize}
    \item 微控制器或者数字信号处理器
    \item 存储器
    \item 提供时间信号的晶振等
    \item 计数器等外设
    \item 标准接口,如串口、网线等
    \item DAC 以及 ADC 完成数模转换
    \item 电源模块维持工作
\end{itemize}
并且可以在外部进行编程控制。

对比传统的传统基于主板的 PC 体系结构形成对比,后者基于功能分离组件。与具有等效功能的多芯片设计相比,集成程度更高的设计可以提高性能并减少功耗以及面积。但是这是以减少组件的\textbf{可替换性}为代价的。

SoC 广泛的应用到嵌入式系统、移动计算、计算机等。

\homep{什么是Top-Down设计模式?优缺点是什么?}

即自顶向下的设计方式,从系统级设计出发,逐步深入到行为级与 RTL 级设计,并对其进行仿真、综合、后仿真的迭代逐渐满足设计需求。

优点是全局出发,逐渐分析需求,将系统分割为子系统,递归应用此过程直到不可分割,可以从全局的角度对设计进行评估,加快了设计的流程,并且提高了模块复用性,可以应用于超大型的集成电路设计。

缺点是需要大量的 EDA 工具的辅助,而不是仅仅通过原理图以及版图进行辅助,对 EDA 的要求很高,并且不能预先估测设计的芯片子系统的物理属性即面积功耗等。

\homep{数字系统设计过程中,“设计说明书(Specification, Spec)”的作用是什么?通常包含哪些内容?}

Spec 是一种架构层次的定义, 可以用于沟通 SoC 设计者、PCB 厂家以及消费者。对芯片解决的需求进行明确,对功能进行定义,对模块进行划分,并且提出可行的解决方案和后续的维护方案。

包括:

\begin{itemize}
    \item 需求 spec
    \item 功能 spec
    \item 设计或产品实现 spec
    \item 维护方案 spec
    % \item datasheet spec 
\end{itemize}
% End Here

\end{document}
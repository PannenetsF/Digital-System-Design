\documentclass[lang=cn,11pt,a4paper,cite=authoryear]{elegantpaper}

% 微分号
\newcommand{\dd}[1]{\mathrm{d}#1}
\newcommand{\pp}[1]{\partial{}#1}

% FT LT ZT
\newcommand{\ft}[1]{\mathscr{F}[#1]}
\newcommand{\fta}{\xrightarrow{\mathscr{F}}}
\newcommand{\lt}[1]{\mathscr{L}[#1]}
\newcommand{\lta}{\xrightarrow{\mathscr{L}}}
\newcommand{\zt}[1]{\mathscr{Z}[#1]}
\newcommand{\zta}{\xrightarrow{\mathscr{Z}}}

% 积分求和号

\newcommand{\dsum}{\displaystyle\sum}
\newcommand{\aint}{\int_{-\infty}^{+\infty}}

% 简易图片插入
\newcommand{\qfig}[3][nolabel]{
  \begin{figure}[!htb]
      \centering
      \includegraphics[width=0.6\textwidth]{#2}
      \caption{#3}
      \label{\chapname :#1}
  \end{figure}
}

% 表格
\renewcommand\arraystretch{1.5}

% 日期

\newcommand{\homep}[1]{\section*{Problem #1}}

\title{数字电路高层次综合设计\quad 第十一周作业}
\author{范云潜 18373486}
\institute{微电子学院 184111 班}
\date{\zhtoday}

\begin{document}

\maketitle

% 作业内容:

\tableofcontents

\listoffigures

本次作业很明显的体现了代码复用的原则,在之后开始一周的作业时,应先对复用性进行一定的分析。本次报告中将模块的定义与端口通过伪代码形式给出。

\section{N 位通用移位寄存器}

对本实验进行模块划分:

\begin{itemize}
    \item 数据移位寄存器: \lstinline{shift (clk_key, data_init, ctrl, data_out)} 
    \item 数据显示七段管: \lstinline{dec (data, HEX)}
\end{itemize}

需要注意到, Quartus 对时钟与复位的要求是,触发类型相同,不能出现电平触发和沿触发混合的形式。对 \lstinline{ctrl} 的控制通过 \lstinline{case} 进行选择即可。



\begin{figure}
    \centering
    \caption{移位寄存器结果}\label{01} 
    % line 1
    \sfig[0.2]{hw11011.jpg} \hfill 
    \sfig[0.2]{hw11012.jpg} \hfill
    \sfig[0.2]{hw11013.jpg} \hfill 
    \sfig[0.2]{hw11014.jpg} \bigskip 
\end{figure}

\section{手动模 M 计数器}

对本实验进行模块划分:

\begin{itemize}
    \item 计数器: \lstinline{cnt (clk, reset, sum)} ,对给定信号的上升沿进行计数,并且设定计数上限为 \lstinline{M} 即可
    \item 使能信号: 可以添加到上一模块,或者将上一模块的输入修改为 \lstinline{clk & en} 
    \item 显示模块: 将计数器的 \lstinline{sum} 输入到 BCD 编码器,将 BCD 编码器的输出接到七段管
\end{itemize}

需要注意不能使用除法以及取模操作,消耗资源极大。



\begin{figure}
    \centering
    \caption{取模器结果}\label{02} 
    % line 1
    \sfig[0.18]{hw11021.jpg} \hfill 
    \sfig[0.18]{hw11022.jpg} \hfill 
    \sfig[0.18]{hw11023.jpg} \hfill 
    \sfig[0.18]{hw11024.jpg} \hfill 
    \sfig[0.18]{hw11025.jpg} \hfill 
\end{figure}

\section{分频器}

首先明确,晶振和需要的输出之间频率的数量级差距很大,因此无需过分在意奇偶分频的区别,通过计数上升沿进行计算的误差会很小。

模块划分:

\begin{itemize}
    \item 分频器\footnote{类似上一实验中的计数器,但是输出变成一位的 \lstinline{tik} 信号}: \lstinline{tik #(.BASE_FREQ(), .NEEDED_FREQ()) (clk, reset, tik_clk)} 
    \item 计数器:\lstinline{cnt (clk, reset, sum)} ,对给定信号的上升沿进行计数,并且设定计数上限为 \lstinline{M} 即可,对分频后的时钟信号进行计数
    \item 显示模块: 将计数器的 \lstinline{sum} 输入到 BCD 编码器,将 BCD 编码器的输出接到七段管
\end{itemize}


\begin{figure}
    \centering
    \caption{分频器结果}\label{03} 
    % line 1
    \sfig[0.2]{hw11031.jpg} \hfill 
    \sfig[0.2]{hw11032.jpg} \hfill
    \sfig[0.2]{hw11033.jpg} \hfill 
    \sfig[0.2]{hw11034.jpg} \bigskip 
\end{figure}

\section{简易电子时钟}


模块划分:

\begin{itemize}
    \item 分频器:\lstinline{tik #(.BASE_FREQ(), .NEEDED_FREQ()) (clk, reset, tik_clk)}  使用上一实验中的分频器即可,获得毫秒周期的信号
    \item 时序逻辑计数:对毫秒进行计数,并且手动产生秒与分钟的进位
    \item BCD 编码器:将之前实验中的 BCD 输入修改为 10 位,来满足 \lstinline{0-1000} 范围内的编码
    \item 显示模块: 将计数器的毫秒、秒、分钟输入到 BCD 编码器,将 BCD 编码器的输出接到七段管
\end{itemize}



\begin{figure}
    \centering
    \caption{数字时钟结果}\label{04} 
    % line 1
    \sfig[0.2]{hw11041.jpg} \hfill 
    \sfig[0.2]{hw11042.jpg} \hfill
    \sfig[0.2]{hw11043.jpg} \hfill 
    \sfig[0.2]{hw11044.jpg} \bigskip 
\end{figure}

\section{7 段管循环显示}

模块划分:

\begin{itemize}
    \item 循环移位寄存器:设定以四位为单元的移位寄存器,将长度与需要的初始值写入参数:

    \lstinline{shift #(.N(), .[4*N-1:0]INIT()) (clk, reset, out)} 
    \item LED 显示: 通过组合逻辑写出
    \item 分频器:产生一秒的时钟信号
    \item 显示模块:需要定义新字符的段码
\end{itemize}

将字符串 \lstinline{'HELLO BUAA'} 进行分析,其中不存在 \lstinline{'HELUA'} ,其他的可以从对应关系中导出: \lstinline{0->U, 8->B, ' '->default} ,恰好补全十进制七段管显示的 \lstinline{0xa-0xe} 。

\begin{figure}
    \centering
    \caption{循环显示结果}\label{05} 
    \sfig[0.22]{hw11051.jpg} \hfill 
    \sfig[0.22]{hw11052.jpg} \hfill 
    \sfig[0.22]{hw11053.jpg} \hfill 
    \sfig[0.22]{hw11054.jpg} 
    
    \bigskip

    \sfig[0.22]{hw11055.jpg} \hfill 
    \sfig[0.22]{hw11056.jpg} \hfill 
    \sfig[0.22]{hw11057.jpg} \hfill 
    \sfig[0.22]{hw11058.jpg} 
    
    \bigskip  
    
    \sfig[0.22]{hw11059.jpg} \hfill 
    \sfig[0.22]{hw110510.jpg} \hfill 
    \sfig[0.22]{hw110511.jpg} \hfill 
\end{figure}


\end{document}
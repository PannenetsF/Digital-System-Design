\documentclass[lang=cn,11pt,a4paper,cite=authoryear]{elegantpaper}

% 微分号
\newcommand{\dd}[1]{\mathrm{d}#1}
\newcommand{\pp}[1]{\partial{}#1}

% FT LT ZT
\newcommand{\ft}[1]{\mathscr{F}[#1]}
\newcommand{\fta}{\xrightarrow{\mathscr{F}}}
\newcommand{\lt}[1]{\mathscr{L}[#1]}
\newcommand{\lta}{\xrightarrow{\mathscr{L}}}
\newcommand{\zt}[1]{\mathscr{Z}[#1]}
\newcommand{\zta}{\xrightarrow{\mathscr{Z}}}

% 积分求和号

\newcommand{\dsum}{\displaystyle\sum}
\newcommand{\aint}{\int_{-\infty}^{+\infty}}

% 简易图片插入
\newcommand{\qfig}[3][nolabel]{
  \begin{figure}[!htb]
      \centering
      \includegraphics[width=0.6\textwidth]{#2}
      \caption{#3}
      \label{\chapname :#1}
  \end{figure}
}

% 表格
\renewcommand\arraystretch{1.5}

% 日期

\newcommand{\homep}[1]{\section*{Problem #1}}

\title{数字电路高层次综合设计\quad 第三周作业}
\author{范云潜 18373486}
\institute{微电子学院 184111 班}
\date{\zhtoday}

\begin{document}

\maketitle

% \tableofcontents
% 如果您看到这句话的话,请允许我提出一些建议,事实上我认为这样的模块(无复杂功能与复杂逻辑),可以将所有报告合并到一处,也就是我提供的 all.pdf ,因为这样的设计没有体现设计者的思考,只是简单的应用语法。如果可能的话,您可以参考这个网站 https://hdlbits.01xz.net/wiki/Main_Page 来实现同学们对语法的认识,对 testbench 进行系统的说明。
% Start Here

\section{设计与测试综述}

本节对实验的设计方法和测试方法进行说明,在具体模块的实现后不再赘述。对单个模块进行系统层次的介绍,即输入、输出以及数据关系的说明。

\subsection{设计方法}

对于结构级设计方法或者说从门级进行设计的方法,产生的硬件有着最高的确定性,但是设计较困难。在本实验设计的简单逻辑中仍可接受,若是与复杂计算逻辑相关,本这样的方法很难迅速的实现,并且最终结果也缺乏可读性,难以修改。

对于数据流式的设计方法,设计思路直观,设计迅速,但是底层实现不能确定,和工艺以及目标平台相关。这是在较高层次的设计中比较常用的方式。

在本次实验中仅涉及到组合逻辑,我们对于输入的下一节点进行分组,可以得到不同节点的计算式,重复应用直到输出级即可完成设计。

\subsection{测试方法}

对于所有的模块的 TestBench 均采用 Golden Model 方式进行测试,通过 Python 生成多组随机数据作为输入,计算后得到输出,将输入输出写入文件。在 TestBench 中通过 \lstinline{readmemh/readmemb} 读取到测试 Buf 中,逐个输入后比对输出,若是输出错误将信息 \lstinline{display} 到标准输出便于调试。同时,将所有的变量的波形进行存储,便于调试。需要注意的是,测试应该使用逻辑全等也就是 \lstinline{!==, ===} 来进行测试,防止悬空引脚的高阻态全等。

对于同一模块不同实现可以使用同样的,通过 \lstinline{`define, `ifdef} 等实例化不同的模块。

对大型的工程应该使用 makefile 等方式对测试单元进行管理,如引用库等,但是本次实验仅涉及到简单的逻辑模块调用,无需如此管理。

\section{7458 芯片}

系统输入:两组三位输入与两组二位输入;系统输出:两组单位输出;系统功能:对两组相同位宽的输入进行与操作,再将两组的结果进行或操作,分别输出。

按照芯片的门级表示进行门级元件调用(结构级)或者通过 \lstinline{assign} 或 \lstinline{always} (数据流)进行两种方式的实现。

\section{向量逆序输出}

系统输入:一个 32 bit 或 4 byte 的数据;系统输出:一个 32 bit 或 4 byte 的数据;系统功能:对 byte 顺序进行倒置,如 \{b1, b2, b3, b4\} 会引起 \{b4, b3, b2, b1\} 的输出(每一个 b* 都代表一个 8 bit 数据)。

通过 Verilog 的位选语法进行分组与重组。

\section{加法器}

系统输入:两个加数,一个进位;系统输出:一个与加数同样位宽的和,一个进位;系统功能:对两个加数和进位进行加和,将产生的和的主体与进位输出。

对于 16 位加法器的两个加数和一个进位通过 \lstinline{assign} 完成数据加法,输出到 \lstinline{cout,sum}

对于 32 位的加法器,需要由 16 位的加法器进行串联,第一级的进位为第二级的输入,进行模块调用即可。

% End Here

\end{document}